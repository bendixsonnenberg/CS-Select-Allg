\documentclass[a4paper]{scrreprt}
 
\usepackage[german]{babel}
\usepackage[utf8]{inputenc}
\usepackage[T1]{fontenc}
\usepackage{ae}
\usepackage[bookmarks,bookmarksnumbered]{hyperref}
\usepackage{glossaries}

\makeglossaries
\newglossaryentry{Produkt}
{
    name=Produkt,
    description={ das von uns gelieferte Softwaresystem}
}

\newglossaryentry{Webbrowser} 
{
    name=Webbrowser,
    description={Für dieses Produkt wird nur auf Google Chrome und Mozilla Firefox hin entwickelt.}
}
\newglossaryentry{Spiel}
{   
    name=Spiel,
    description={Ein Spiel ist eine Instanz eines Spielmodus. Ein Spiel hat das Ziel das Wissen des Spielers zu nutzen um die Merkmalsauswahl für Machine Learning zu unterstützen.}
}
\newglossaryentry{Spieler}
{
    name=Spieler,
    description={Ein Spieler ist ein Nutzer, welcher an einem Spiel teilnehmen kann. Meist ist dies ein Angestellter des Betriebs.}
}
\begin{document}
 
\title{Pflichtenheft CS-Select}
\author{PSEArmee}
\maketitle
 
% Platzierung des Inhaltsverzeichnisses
\tableofcontents
 
\chapter{Zielbestimmung}
Eine Organisation soll durch das \Gls{Produkt} das Domänenwissen ihrer Mitarbeiter dazu nutzten die Merksmalsauswahl für ein Machiene-Learning-Modell zu vereinfachen. 
\section{Musskriterien}
\begin{tabular}{ l | r}
/FA10/ & Ein \Gls{Spieler} muss sich anmelden können. \\ 
/FA20/ & Ein \Gls{Spieler} muss sich registrieren können. \\
/FA30/ & Ein Organisator muss sich anmelden können. \\
/FA40/ & Ein Organisator muss ein \Gls{Spiel} erstellen können. \\
/FA50/ & Die Anmeldung muss in einem modernem \Gls{Webbrowser} möglich sein. \\
/FA60/ & \Gls{Spieler} müssen bei einem \Gls{Spiel} mitspielen können. \\
\end{tabular} 
\section{Kannkriterien}
\begin{itemize}
\item 1
\end{itemize} 
 
\section{Abgrenzungskriterien}
\begin{itemize} 
\item 1
\end{itemize} 
 
\chapter{Einsatz}
 
\section{Anwendungsbereiche}
 
\section{Zielgruppen}
 
\section{Betriebsbedingungen}
 
\chapter{Umgebung}
 
\section{Software}
 
\section{Hardware}
 
\chapter{Produktdaten}

\chapter{Systemmodelle}
\section{Szenarien}
\section{Anwendungsfälle}
\section{Objektmodell}
\section{Dynamische Modelle}
\section{Benutzerschnittstelle}
\clearpage
\chapter{Glossar}
\printglossary 
\end{document}
