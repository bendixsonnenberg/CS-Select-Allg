\documentclass[a4paper]{scrreprt}
 
\usepackage[german]{babel}
\usepackage[utf8]{inputenc}
\usepackage[T1]{fontenc}
\usepackage{ae}
\usepackage[bookmarks,bookmarksnumbered]{hyperref}
\usepackage{glossaries}

\makeglossaries
\newglossaryentry{Produkt}
{
    name=Produkt,
    description={ das von uns gelieferte Softwaresystem}
}

\begin{document}
 
\title{Pflichtenheft CS-Select}
\author{PSEArmee}
\maketitle
 
% Platzierung des Inhaltsverzeichnisses
\tableofcontents
 
\chapter{Zielbestimmung}
Eine Organisation soll durch das \Gls{Produkt} das Domänenwissen ihrer Mitarbeiter dazu nutzten die Merksmalsauswahl für ein Machiene-Learning-Modell zu vereinfachen. 
\section{Musskriterien}
\begin{itemize}
\item 1
\end{itemize} 
\section{Kannkriterien}
\begin{itemize}
\item 1
\end{itemize} 
 
\section{Abgrenzungskriterien}
\begin{itemize} 
\item 1
\end{itemize} 
 
\chapter{Einsatz}
 
\section{Anwendungsbereiche}
 
\section{Zielgruppen}
 
\section{Betriebsbedingungen}
 
\chapter{Umgebung}
 
\section{Software}
 
\section{Hardware}
 
\chapter{Produktdaten}

\chapter{Systemmodelle}
\section{Szenarien}
\section{Anwendungsfälle}
\section{Objektmodell}
\section{Dynamische Modelle}
\section{Benutzerschnittstelle}
\clearpage
\chapter{Glossar}
\printglossary 
\end{document}
