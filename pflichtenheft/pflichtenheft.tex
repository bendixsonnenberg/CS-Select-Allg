\documentclass[a4paper]{scrreprt}
 
\usepackage[german]{babel}
\usepackage[utf8]{inputenc}
\usepackage[T1]{fontenc}
\usepackage{ae}
\usepackage[bookmarks,bookmarksnumbered]{hyperref}
\usepackage{glossaries}
\usepackage{tabularx}
\usepackage{graphicx}

\makeglossaries
\newglossaryentry{Produkt}
{
    name=Produkt,
    description={ das von uns gelieferte Softwaresystem}
}
\newglossaryentry{Webbrowser} 
{
    name=Webbrowser,
    description={Für dieses Produkt wird nur auf Google Chrome und Mozilla Firefox hin entwickelt.}
}
\newglossaryentry{Spiel}
{   
    name=Spiel,
    description={Ein Spiel ist eine Instanz eines Spielmodus. Ein Spiel hat das Ziel das Wissen des Spielers zu nutzen um die Merkmalsauswahl für Machine Learning zu unterstützen.}
}
\newglossaryentry{Spieler}
{
    name=Spieler,
    description={Ein Spieler ist ein Nutzer, welcher an einem Spiel teilnehmen kann. Meist ist dies ein Angestellter des Betriebs.}
}
\newglossaryentry{Spieleinstellungen}
{
    name=Spieleinstellungen,
    description={Einstelllungen für ein Spiel umfassen: die Merkmale welche ausgewählt werden soll, die Art des Spiels, die teilnehmenden Spieler, Endbedingungen des Spiels}
}
\newglossaryentry{Organisator}
{
    name=Organisator,
    description={Ein Organisator ist ein Nutzer, der neue Spiele erstellt und die Ergebnisse von diesen ausliest.}
}
\newglossaryentry{Spiele}
{
    name=Spiele,
    description={Plural von \Gls{Spiel}}
}
\newglossaryentry{Achievement}
{  
    name=Achievement,
    description={Ein Ziel oder eine Errungenschaft, welche den Spieler motiviert weiter zu spielen.}
}
\newglossaryentry{Administrator}
{
    name=Administrator,
    description={Die Person, welche das System installiert und den Nutzern zur Verfügung stellt. Sie verwaltet den
                \Gls{Spiele-Server}}
}
\newglossaryentry{Spiele-Server}
{
name=Spiele-Server,
description={Ein Computer, der der Verwaltung von CS:Select dient und Internetanbindung hat.}
}

\begin{document}
 
\title{Pflichtenheft CS-Select}
\author{PSEArmee}
\maketitle
 
% Platzierung des Inhaltsverzeichnisses
\tableofcontents
 
\chapter{Zielbestimmung}
Eine Organisation soll durch das \Gls{Produkt} das Domänenwissen ihrer Mitarbeiter dazu nutzen die Merksmalsauswahl für ein Machine-Learning-Modell zu vereinfachen.

\section{Musskriterien}
\begin{tabular}{ l | l}
/FA10/ & Ein \Gls{Spieler} muss sich anmelden können. \\ 
/FA20/ & Ein \Gls{Spieler} muss sich registrieren können. \\
/FA30/ & Ein \Gls{Organisator} muss sich anmelden können. \\
/FA40/ & Ein \Gls{Organisator}\Gls{Organisator} muss ein \Gls{Spiel} erstellen und beenden können. \\
/FA50/ & Die Anmeldung muss in einem modernem \Gls{Webbrowser} möglich sein. \\
/FA60/ & \Gls{Spieler} müssen bei einem \Gls{Spiel} mitspielen können. \\
/FA70/ & Ein \Gls{Organisator} muss \Gls{Spieler} zu einem \Gls{Spiel} einladen. \\
\end{tabular}

\section{Kannkriterien}
\begin{tabularx}{\linewidth}{@{}>{\bfseries}l@{\hspace{.5em}}X@{}} % Linebreaks in der Tabelle
/KA10/ & Ein \Gls{Organisator} kann die Ergebnisse/Eingabe eines Spielers ansehen. \\
/KA20/ & Der \Gls{Administrator} kann Spiele aus dem System löschen. \\
/KA30/ & Der \Gls{Organisator} kann Spiele löschen. \\
/KA40/ & Ein \Gls{Organisator} kann eines seiner \Gls{Spiele} löschen. \\
/KA50/ & Der \Gls{Spieleserver} kann vom Terminal aus beendet werden. \\
/KA60/ & Beim Erstellen eines Spiels kann der \Gls{Organisator} die \Gls{Spieleinstellungen} speichern und laden. \\
/KA70/ & Das System kann auf verschiedenen Ports ausgeführt werden. \\
/KA80/ & Beim Start des Systems erscheint ein Dialog zum Einrichten des Systems. \\
/KA90/ & Die Ergebnisse eines \Gls{Spiel}s werden im Interface des \Gls{Organisator}s angezeigt. \\
/KA100/ & Beim Spielen existiert ein Knopf, welcher das Überspringen der aktuellen Runde ermöglicht. \\
/KA110/ & Der Knopf aus /KA100/ kostet den \Gls{Spieler} Punkte. \\ %Kostet der Knopf Punkte? Oder das Betätigen
/KA120/ & Es ist \Gls{Spieler}n möglich Kriterien auszuschließen, also diese als unwichtig zu markieren. \\
/KA130/ & Der \Gls{Organisator} eines Spiels kann nach dem Begin eines Spiels noch weitere Spieler einladen. \\
/KA140/ & Die Merkmale, die pro Runde angezeigt werden, sind schlauer ausgewählt als durch Zufall. \\ %Schlauer präzisieren?
/KA150/ & Die maximale Anzahl an Runden pro Spieler pro Tag können durch den \Gls{Organisator} eingeschränkt werden. \\
/KA160/ & \Gls{Spieler} erhalten mehr Punkte, wenn sie mehrere Runden in Folge spielen. \\
/KA170/ & Es gibt \Gls{Achievement}s, welche von Spielern freigeschaltet werden. \\
/KA180/ & Es gibt eine Auflistung der \Gls{Achievement}s aus /KA170/. \\
/KA190/ & Es gibt einen Hilfe Button im \Gls{Spieler} GUI. \\ %Knopf anstatt Button?
/KA200/ & Es gibt einen Hilfe Button im \Gls{Organisator} GUI. \\ %s. o.
/KA210/ & Das Webinterface unterstüzt Internet Explorer. \\
/KA220/ & Mehrere Sprachen sind einfach zu integrieren. \\ 
/KA230/ & Es ist möglich die Punkte eines \Gls{Spieler}s zurückzusetzten. \\ %Durch Spieler oder Orga?
/KA240/ & Es gibt ein GUI für \Gls{Administrator}en. \\ %Die oder der Gui? ^^
\end{tabularx}

\section{Abgrenzungskriterien}
\begin{itemize} 
  \item 1
\end{itemize} 
 
\chapter{Einsatz}

\section{Anwendungsbereiche}
Das \Gls{Produkt} dient der Verbesserung der Merkmalsauswahl bei Machine-Learning Prozessen in wissenschaftlichen
Experimenten beziehungsweise privatwirtschaftlichen Unternehmen durch das Domänenwissen der \Gls{Spieler}.

\section{Zielgruppen}
Die Zielgruppen des \Gls{Produkt}s lassen sich in \Gls{Organisator}en und \Gls{Spieler} unterscheiden.
Der Organisator möchte eine Verbesserung der Machine-Learning Prozesse erreichen indem er das Domänenwissen der Spieler nutzt.
Die Spieler tragen durch Spielen des \Gls{Produkt}s mit ihrem Domänenwissen zur besseren Merkmalsauswahl bei.

\section{Betriebsbedingungen}
Das \Gls{Produkt} ist für die Nutzung in Büroräumlichkeiten vorgesehen.

\chapter{Umgebung}
Das \Gls{Produkt} läuft auf einem Server, die Nutzung erfolgt an Arbeitsplatzrechnern.

\section{Software}
Das \Gls{Produkt} läuft auf Linux ab Kernel Version 4 und Windows 7 oder neuer.

\section{Hardware}
Das \Gls{Produkt} läuft auf Server-Computern.

\chapter{Produktdaten}

\section{Nutzerdaten}
\begin{tabularx}{\linewidth}{@{}>{\bfseries}l@{\hspace{.5em}}X@{}}
/D10/ & Über Nutzer(\Gls{Spieler}, \Gls{Organisator}) sind folgende Daten zu speichern /ND10/: \begin{itemize}
	\item Email-Adresse, Nutzername, Punktestand, Rolle, Passwort(als Hash)
\end{itemize}
\end{tabularx}

\section{Merkmalsdatensätze}
\begin{tabularx}{\linewidth}{@{}>{\bfseries}l@{\hspace{.5em}}X@{}}
	/D20/ & Für jeden Merkmalsdatensatz ist zu speichern: /ND20/: \begin{itemize}
		\item Liste an Merkmalen mit jeweiligen Graphen und Beschreibung
	\end{itemize}
\end{tabularx}

\section{Spieledaten}
\begin{tabularx}{\linewidth}{@{}>{\bfseries}l@{\hspace{.5em}}X@{}}
	/D30/ & Für jedes \Gls{Spiel} ist zu speichern /ND30/: \begin{itemize}
		\item \Gls{Organisator}, \Gls{Spieleinstellungen}, \Gls{Spieler}, zugehöriger Merkmalsdatensatz % zuständiger ML Server?
	\end{itemize}
\end{tabularx}
\chapter{Systemmodelle}
\section{Szenarien}
\section{Anwendungsfälle}
    \subsection{Admin-Account einrichten}
    Beim erstmaligen Aufsetzen des \Gls{Spiele-Server}s kann ein Admin-Account in einer config-Datei festgelegt werden.
    Dazu werden Name, Email und Passwort des \Gls{Administrator}s gespeichert.
    \subsection{Achievments ansehen}
    Nach der Anmeldung erreicht ein \Gls{Spieler} eine Übersichts-Seite. Beim Klicken auf die Spieler-Übersicht-Schaltfläche
    (in \fref{chap:Spieler-Übersicht} gezeigt) kann der \Gls{Spieler} seine Errungenschaften einsehen.
    \subsection{Leaderboard ansehen}
    Nach der Anmeldung erreicht ein \Gls{Spieler} eine Übersichts-Seite. Darauf ist ein Bereich, der eine aktuelle Tabelle
    anzeigt, in welcher die \Gls{Spieler} aufgelistet sind, die in einem bestimmten Zeitraum die meisten Punkte erreicht haben.

\section{Objektmodell}
\section{Dynamische Modelle}
\section{Benutzerschnittstelle}
    \subsection{Anmeldung}
        \centering
        \includegraphics[width=400px]{../pictures/1_Anmeldung.jpg}
    \subsection{Organisator-Start (Desktop)}
        \centering
        \includegraphics[width=400px]{../pictures/2_Organisator.jpg}
    \subsection{Organisator-Start (Responsive)}
        \centering
        \includegraphics[width=400px]{../pictures/3_Organisator(responsive).jpg}
    \subsection{Spielerstellung}
        \centering
        \includegraphics[width=400px]{../pictures/4_Spielerstellung.jpg}
    \subsection{Spieler-Übersicht}
        \centering
        \includegraphics[width=400px]{../pictures/5_Spieler.jpg}
    \subsection{Spiel}
        \centering
        \includegraphics[width=400px]{../pictures/6_Spiel.jpg}
\clearpage
\printglossary 
\end{document}
