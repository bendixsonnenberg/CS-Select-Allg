\documentclass[a4paper]{scrreprt}

\usepackage[ngerman]{babel}
\usepackage[utf8]{inputenc}
\usepackage[T1]{fontenc}
\usepackage{ae}
\usepackage[bookmarks, bookmarksnumbered]{hyperref}
\usepackage{tabularx}
\usepackage{graphicx}
\usepackage{csquotes}
\usepackage{verbatim}
\usepackage{float}
\usepackage[nonumberlist, toc, section]{glossaries}
\usepackage[german]{fancyref}

\setcounter{secnumdepth}{4}

\begin{document}

    \title{Qualitätssicherungsdokument CS:Select}
    \author{Luca Springer, Alexander Linder, Julian Dinh, Nicholas Bieker,\\ Bendix Sonnenberg}
    \maketitle

    \tableofcontents
    \chapter{Einleitung}
    Die Qualitätssicherung ist die abschließende Phase unseres Projekts CS:Select. In dieser Phase wurde das Produkt
    gründlich und umfangreich getestet. Jegliche Fehler, welche nach der Implementierungsphase geblieben sind, sollten einerseits
    gefunden und andererseits beseitigt werden. Außerdem ging es darum,
    die Code-Qualität noch weiter zu verbessern und auch schon vorhandene Unit-Tests weiter auszubauen. \\
    \\
    In diesem Dokument befassen wir uns insbesondere damit, wie wir beim Testen vorgegangen sind. Als Grundlage dazu
    dienen die Testfallszenarien aus dem Pflichtenheft, welche wir abgearbeitet haben, aber auch eigene Testfälle, die wir hinzugefügt haben. Außerdem
    haben auch andere Personen, welche nicht an dem Produkt gearbeitet haben, das Produkt ausprobiert und uns Feedback gegeben. Daraufhin
    konnten wir das Produkt noch weiter verbessern, z.B. wenn die Funktionsweise an manchen Stellen nicht intuitiv genug für den Nutzer war.
    Insgesamt haben wir in dieser Phase also auch die Nutzerfreundlichkeit erhöht und weitere Features eingebaut, die das Nutzen der Anwendung erleichtern. \\
    \\
    Als weiteren Punkt beschreiben wir unsere Testüberdeckungswerte, wie diese erreicht wurden und vergleichen diese noch mit den Tests aus der Implementierung. \\
    \\
    Schließlich sind durch all diese Testverfahren diverse Fehler erkannt und beseitigt worden. Im letzten Teil des Dokuments
    reflektieren wir noch einmal diese aufgetretenen Fehler. Dabei geht es auch darum, warum sie überhaupt auftraten
    und wie sie von uns behoben worden sind.





    \chapter{Testfälle}

        \section{Pflichtenheftszenarien}
        Die folgenden Szenarien sind alle dem Pflichtenheft entnommen (siehe Referenz in Überschriften). Sie sind mit dem deutschen Sprachpaket durchgeführt worden.

            \subsection{Spieler registrieren (8.1.1)}
            \begin{itemize}
                \item Ausgangssituation: Spieler befindet sich auf der Login-Seite
                \item Schritte:
                    \begin{itemize}
                        \item Auf \enquote{Brauchst du einen Account?} klicken
                        \item Gültige E-Mail-Adresse in das E-Mail-Feld eingeben (\enquote{spieler@csselect.com})
                        \item Beliebiges Passwort in Passwort-Feld eingeben (\enquote{testPasswort})
                        \item Passwort wiederholen (\enquote{testPasswort})
                        \item \enquote{Spieler} auswählen
                        \item Einen Nutzernamen eingeben (\enquote{spieler})
                        \item Auf \enquote{Registrieren} klicken
                    \end{itemize}
                \item Funktioniert: Ja
            \end{itemize}

            \subsection{Spieler anmelden (8.1.2)}
            \begin{itemize}
                \item Ausgangssituation: Spieler befindet sich auf der Login-Seite und besitzt bereits einen Account
                \item Schritte:
                    \begin{itemize}
                        \item Registrierte E-Mail-Adresse in das E-Mail-Feld eingeben (\enquote{spieler@csselect.com})
                        \item Dazugehöriges Passwort in Passwort-Feld eingeben (\enquote{testPasswort})
                        \item Auf \enquote{Anmelden} klicken
                    \end{itemize}
                \item Funktioniert: Ja
                \item Änderungen zum Pflichtenheftszenario
                \begin{itemize}
                    \item Es gibt keine Checkbox für den Spieler, man muss die Organisator-Checkbox leer lassen
                \end{itemize}
            \end{itemize}

            \subsection{Spieler abmelden (8.1.3)}
            \begin{itemize}
                \item Ausgangssituation: Spieler ist angemeldet
                \item Schritte:
                    \begin{itemize}
                        \item In der Navigationsleiste auf \enquote{Abmelden} klicken
                    \end{itemize}
                \item Funktioniert: Ja
            \end{itemize}

            \subsection{Organisator registrieren (8.1.4)}
            \begin{itemize}
                \item Ausgangssituation: Organisator befindet sich auf der Login-Seite
                \item Schritte:
                    \begin{itemize}
                        \item Auf \enquote{Brauchst du einen Account?} klicken
                        \item Gültige E-Mail-Adresse in das E-Mail-Feld eingeben (\enquote{organiser@csselect.com})
                        \item Beliebiges Passwort in Passwort-Feld eingeben (\enquote{testPasswort})
                        \item Passwort wiederholen (\enquote{testPasswort})
                        \item \enquote{Organisator} auswählen
                        \item Das Master-Passwort eingeben (\enquote{sicherespasswort})
                        \item Auf \enquote{Registrieren} klicken
                    \end{itemize}
            \item Funktioniert: Ja
            \end{itemize}

            \subsection{Organisator anmelden (8.1.5)}
            \begin{itemize}
                \item Ausgangssituation: Organisator befindet sich auf der Login-Seite und besitzt bereits einen Account
                \item Schritte:
                    \begin{itemize}
                        \item Registrierte E-Mail-Adresse in das E-Mail-Feld eingeben (\enquote{spieler@csselect.com})
                        \item Dazugehöriges Passwort in Passwort-Feld eingeben (\enquote{testPasswort})
                        \item \enquote{Organisator} auswählen
                        \item Auf \enquote{Anmelden} klicken
                    \end{itemize}
                \item Funktioniert: Ja
            \end{itemize}

            \subsection{Organisator abmelden (8.1.6)}
            \begin{itemize}
                \item Ausgangssituation: Organisator ist angemeldet
                \item Schritte:
                    \begin{itemize}
                        \item In der Navigationsleiste auf \enquote{Abmelden} klicken
                    \end{itemize}
                \item Funktioniert: Ja
            \end{itemize}

            \subsection{Organisator löschen (8.1.7)}
            Da die Kommunikation eines Administrators mit dem System über eine Kommandozeile nicht möglich ist, kann dieses Szenario nicht getestet werden.
            Das Löschen eines Organisators kann jedoch direkt in der Datenbank durch den Administrator erfolgen.

            \subsection{Spiel erstellen (8.1.8)}
            \begin{itemize}
                \item Ausgangssituation: Organisator ist angemeldet
                \item Schritte:
                    \begin{itemize}
                        \item In der Navigationsleiste auf \enquote{Spielerstellung} klicken
                        \item Mindestens alle Pflichtfelder korrekt ausfüllen
                        \item Titel (\enquote{TestSpiel})
                        \item Beschreibung (\enquote{Dieses Spiel ist ein Test.})
                        \item Spielmodus (\enquote{BinarySelect})
                        \item Terminierung (\enquote{Organisator Terminierung})
                        \item Feature Set (\enquote{populationGender})
                        \item Datenbankname (\enquote{Datenbank})
                        \item Auf \enquote{Erstellen} klicken
                    \end{itemize}
                \item Funktioniert: Ja
            \end{itemize}

            \subsection{Spieler nach Spielbeginn einladen (8.1.9)}
            \begin{itemize}
                \item Ausgangssituation: Organisator ist angemeldet
                \item Schritte:
                    \begin{itemize}
                        \item Auf der Übersichts-Seite das entsprechende Spiel suchen und auf \enquote{Einladen} klicken
                        \item Gültige E-Mail-Adresse in das Feld eingeben
                        \item Alternativ auf \enquote{Mehrere Spieler einladen} klicken und mehrere E-Mails, getrennt mit Kommata, eingeben
                        \item Auf \enquote{Einladen} klicken
                    \end{itemize}
                \item Funktioniert: Ja
            \end{itemize}

            \subsection{Spiel vorzeitig beenden (8.1.10)}
            \begin{itemize}
                \item Ausgangssituation: Organisator ist angemeldet
                \item Schritte:
                    \begin{itemize}
                        \item Auf der Übersichts-Seite das entsprechende Spiel suchen und auf \enquote{Beenden} klicken
                \end{itemize}
                \item Funktioniert: Ja
            \end{itemize}

            \subsection{Spiel aus Übersicht löschen (8.1.11)}
            \begin{itemize}
                \item Ausgangssituation: Organisator ist angemeldet und hat terminierte Spiele
                \item Schritte:
                    \begin{itemize}
                        \item Das entsprechende terminierte Spiel aussuchen und auf \enquote{Löschen} klicken
                        \item Spiel verschwindet aus der Organisator-Übersicht
                \end{itemize}
                \item Funktioniert: Ja
            \end{itemize}

            \subsection{Spieleinstellungen speichern (8.1.12)}
            Für diesen Testfall gibt es zwei Varianten. Man kann eine Vorlage während oder nach der Erstellung eines Spiels speichern. \\

            \textbf{Während der Spielerstellung}
            \begin{itemize}
                \item Ausgangssituation: Organisator ist auf der Spielerstellungs-Seite
                \item Schritte:
                    \begin{itemize}
                        \item Alle Pflichtfelder korrekt ausfüllen
                        \item Checkbox zum Erstellen einer Vorlage setzen
                        \item Im Textfeld \enquote{Name der Vorlage} einen Name für die Vorlage auswählen (\enquote{TestVorlage})
                        \item Auf \enquote{Erstellen} klicken
                    \end{itemize}
                \item Funktioniert: Ja
            \end{itemize}

            \textbf{Nach der Spielerstellung}
            \begin{itemize}
                \item Ausgangssituation: Organisator ist auf der Übersichts-Seite
                \item Schritte:
                    \begin{itemize}
                        \item
                        \item
                        \item
                        \item
                \end{itemize}
                \item Funktioniert: Ja
            \end{itemize}

            \subsection{Spieleinstellungen laden (8.1.13)}
            \begin{itemize}
                \item Ausgangssituation: Organisator ist auf der Spielerstellungs-Seite und eine Voreinstellung exisitiert bereits
                \item Schritte:
                    \begin{itemize}
                        \item Bei \enquote{Voreinstellung} eine gewünschte Vorlage wählen
                        \item Gegebenenfalls noch Parameter ändern und auf \enquote{Erstellen} klicken
                    \end{itemize}
                \item Funktioniert: Ja
                \item Änderungen zum Pflichtenheftszenario
                \begin{itemize}
                    \item Eine Voreinstellung kann nur bei der Spielerstellung gespeichert werden, aus einem bereits erstellten Spiel kann keine Vorlage erzeugt werden
                \end{itemize}
            \end{itemize}

            \subsection{Spielzustand auslesen (8.1.14)}
            \begin{itemize}
                \item Ausgangssituation: Organisator ist angemeldet
                \item Schritte:
                    \begin{itemize}
                        \item Unter \enquote{Laufende Spiele} werden zu jedem Spiel entsprechende Informationen angezeigt
                    \end{itemize}
                \item Funktioniert: Ja
            \end{itemize}

            \subsection{Einladung zu Spiel annehmen (8.2.1)}
            \begin{itemize}
                \item Ausgangssituation: Spieler ist angemeldet und wurde zu einem Spiel eingeladen
                \item Schritte:
                    \begin{itemize}
                        \item Auf der Übersichts-Seite das gewünschte Spiel aussuchen und auf \enquote{Akzeptieren} klicken
                    \end{itemize}
                \item Funktioniert: Ja
            \end{itemize}

            \subsection{Runde starten (8.3.1)}
            \begin{itemize}
                \item Ausgangssituation: Spieler ist angemeldet und hat mindestens ein aktives Spiel
                \item Schritte:
                    \begin{itemize}
                        \item Auf der Übersichts-Seite das gewünschte Spiel aussuchen und auf \enquote{Spielen} klicken
                    \end{itemize}
                \item Funktioniert: Ja
            \end{itemize}

            \subsection{Runde spielen: Matrix Select (8.2.2)}
            \begin{itemize}
                \item Ausgangssituation: Spieler ist angemeldet und hat offenes Spiel im Spielmodus Matrix Select
                \item Schritte:
                    \begin{itemize}
                        \item Auf Übersichts-Seite das entsprechende Spiel suchen und \enquote{Spielen} wählen
                        \item Bestimmte Merkmale auswählen (innerhalb des angegebenen Intervalls), eventuell auch welche als nutzlos markieren oder Graphen anzeigen lassen
                        \item Auf \enquote{Bestätigen} klicken
                    \end{itemize}
                \item Funktioniert: Ja
            \end{itemize}

            \subsection{Runde spielen: Binär Select (8.2.3)}
            \begin{itemize}
                \item Ausgangssituation: Spieler ist angemeldet und hat offenes Spiel im Spielmodus Binär Select
                \item Schritte:
                    \begin{itemize}
                        \item Auf Übersichts-Seite das entsprechende Spiel suchen und \enquote{Spielen} wählen
                        \item Aus zwei Merkmalen eines wählen, eventuell auch als nutzlos markieren oder Graphen anzeigen lassen
                        \item Vorherigen Schritt fünf Mal durchführen
                        \item Auf \enquote{Bestätigen} klicken
                    \end{itemize}
                \item Funktioniert: Ja
            \end{itemize}

            \subsection{Spielterminierung während Runde (8.2.4)}
            \begin{itemize}
                \item Ausgangssituation: Spieler befindet sich auf der Spiel-Seite von einem Spiel, das gerade die Terminierungsbedingung erreicht hat
                \item Schritte:
                    \begin{itemize}
                        \item Merkmalsauswahl durchführen
                        \item Auf \enquote{Bestätigen} klicken
                        \item Spieler wird mit einer erklärenden Nachricht auf sein Dashboard gebracht
                    \end{itemize}
                \item Funktioniert: Ja
            \end{itemize}

            \subsection{Auswahl überspringen (8.2.5)}
            \begin{itemize}
                \item Ausgangssituation: Spieler befindet sich auf der Spiel-Seite
                \item Schritte:
                    \begin{itemize}
                        \item Gegebenenfalls Merkmale als nutzlos markieren
                        \item Auf \enquote{Überspringen} klicken
                    \end{itemize}
                \item Funktioniert: Ja
                \item Änderungen zum Pflichtenheftszenario
                \begin{itemize}
                    \item Spieler verliert keine Punkte, bekommt aber auch keine
                    \item Streak wird zurückgesetzt, es sei denn, Spieler markiert alle Merkmale als nutzlos
                \end{itemize}
            \end{itemize}

            \subsection{Merkmal als unwichtig markieren (8.2.6)}
            \begin{itemize}
                \item Ausgangssituation: Spieler befindet sich auf der Spiel-Seite
                \item Schritte:
                    \begin{itemize}
                        \item Bei einem oder mehreren unwichtigen Merkmalen auf \enquote{Nutzlos markieren} klicken
                        \item Zusätzlich kann Spieler auch noch Merkmale für seine Merkmalsauswahl aussuchen
                    \end{itemize}
                \item Funktioniert: Ja
            \end{itemize}

            \subsection{Streak erreichen (8.2.7)}
            \begin{itemize}
                \item Ausgangssituation: Spieler befindet sich auf der Spiel-Seite
                \item Schritte:
                    \begin{itemize}
                        \item Mehrere Runden (mind. eine) am Stück spielen
                        \item Im Feld \enquote{Streak} anhand des Balkens und des zugehörigen Textes den Zustand seiner Streak erkennen
                    \end{itemize}
                    \item Funktioniert: Ja
            \end{itemize}

            \subsection{Achievements ansehen (8.3.2)}
            \begin{itemize}
                \item Ausgangssituation: Spieler ist angemeldet
                \item Schritte:
                    \begin{itemize}
                        \item In der Navigationsleiste auf \enquote{Achievements} klicken
                        \item Auf dieser Achievement-Seite die Achievements und deren Status einsehen
                    \end{itemize}
                \item Funktioniert: Ja
            \end{itemize}

            \subsection{Leaderboard ansehen (8.3.3)}
            \begin{itemize}
                \item Ausgangssituation: Spieler ist angemeldet
                \item Schritte:
                    \begin{itemize}
                        \item Direkt auf der Übersichts-Seite die Top 5 einsehen
                    \end{itemize}
                \item Funktioniert: Ja
            \end{itemize}

            \subsection{Sprache ändern (8.4.1)}
            \begin{itemize}
                \item Ausgangssituation: Nutzer ist angemeldet
                \item Schritte:
                    \begin{itemize}
                        \item In der Navigationsleiste auf \enquote{Einstellungen} klicken
                        \item Auf Feld \enquote{Sprache} klicken und anschließend eine der verfügbaren Sprachen auswählen
                        \item Auf \enquote{Aktualisieren} klicken
                    \end{itemize}
                \item Funktioniert: Ja
                \item Anmerkung: Seite muss neu geladen werden, damit die Änderung in Kraft tritt
            \end{itemize}

            \subsection{Server beenden (8.4.3)}
            \begin{itemize}
                \item Ausgangssituation: Server ist aktiviert
                \item Schritte:
                    \begin{itemize}
                        \item Die Tomcat-Befehle \texttt{shutdown.sh} oder \texttt{shutdown.bat} ausführen
                        \item Falls gewollt, den MySQL-Server beenden
                    \end{itemize}
                \item Funktioniert: Ja
            \end{itemize}

            \subsection{Konfiguration bearbeiten (8.4.2)}
            \begin{itemize}
                \item Ausgangssituation: Administrator befindet sich im Installationsverzeichnis des Servers
                \item Schritte:
                    \begin{itemize}
                        \item Konfigurationsdatei mit beliebigem Texteditor öffnen
                        \item Die gewünschten Einstellungen editieren
                        \item Datei speichern
                        \item Server beenden
                        \item Server starten
                    \end{itemize}
                \item Funktioniert: Ja
            \end{itemize}

            \subsection{Passwort ändern (8.4.4)}
            \begin{itemize}
                \item Ausgangssituation: Nutzer ist angemeldet
                \item Schritte:
                    \begin{itemize}
                        \item In der Navigationsleiste auf \enquote{Einstellungen} klicken
                        \item In das Textfeld \enquote{Neues Passwort} das neue Passwort schreiben
                        \item Neues Passwort im Textfeld \enquote{Passwort wiederholen} wiederholen
                        \item Auf \enquote{Aktualisieren} klicken
                    \end{itemize}
                \item Funktioniert: Ja
                \item Änderungen zum Pflichtenheftszenario
                \begin{itemize}
                    \item Das aktuelle Passwort muss nicht angegeben werden
                \end{itemize}
            \end{itemize}

            \subsection{E-Mail ändern (8.4.5)}
            \begin{itemize}
                \item Ausgangssituation: Nutzer ist angemeldet
                \item Schritte:
                    \begin{itemize}
                        \item In der Navigationsleiste auf \enquote{Einstellungen} klicken
                        \item In das Textfeld \enquote{Neue E-Mail} eine neue gültige E-Mail einfügen
                        \item Auf \enquote{Aktualisieren} klicken
                    \end{itemize}
                \item Funktioniert: Ja
                \item Änderungen zum Pflichtenheftszenario
                    \begin{itemize}
                        \item Das aktuelle Passwort muss nicht angegeben werden
                    \end{itemize}
            \end{itemize}

            \subsection{Passwort zurücksetzen (8.4.6)}
            \begin{itemize}
                \item Ausgangssituation: Nutzer befindet sich auf der Login-Seite
                \item Schritte:
                    \begin{itemize}
                        \item Auf \enquote{Du hast dein Passwort vergessen?} klicken
                        \item Seine E-Mail in das E-Mail-Feld einfügen
                        \item Falls man Organisator ist: \enquote{Organisator} auswählen
                        \item Auf \enquote{Passwort zurücksetzen} klicken
                        \item Mit dem per E-Mail erhaltenen temporären Passwort anmelden
                    \end{itemize}
                \item Funktioniert: Ja
            \end{itemize}

            \subsection{Hilfe-Schaltfläche verwenden (8.4.7)}
            Dieser Testfall kann nicht durchgeführt werden. Es existieren keine expliziten Hilfe-Schaltflächen, denn diese würden auf der GUI eher für Verwirrung sorgen.
            Stattdessen gibt es zahlreiche Tooltips und ein Organistor hat beispielsweise einen Hilfstext. Diese Features sollen die Bedienung erleichtern und eine Alternative für Hilfe-Schaltflächen darstellen.


        \section{Weitere Testfallszenarien}
        Die folgenden Szenarien sind alle in der Qualitätssicherungphase entstanden und von uns erstellt worden. Sie sind ebenso mit dem deutschen Sprachpaket durchgeführt worden.

            \subsection{Nutzer bei Inaktivität abmelden}
            \begin{itemize}
                \item Ausgangssituation: Nutzer ist angemeldet
                \item Schritte:
                    \begin{itemize}
                        \item Für zehn Minuten nichts im Browser-Fenster tun
                    \end{itemize}
                \item Funktioniert: Ja
            \end{itemize}

            \subsection{Daily-Challenge abschließen}
            \begin{itemize}
                \item Ausgangssituation: Spieler ist angemeldet
                \item Schritte:
                    \begin{itemize}
                        \item Auf der Übersichts-Seite unter \enquote{Tägliche Aufgabe} die Daily-Challenge einsehen
                        \item Die Aufgabe erfüllen (z.B. drei Runden spielen)
                        \item Daily-Challenge abgeschlossen und Punkte erhalten
                    \end{itemize}
                \item Funktioniert: Ja
            \end{itemize}


    \chapter{Testmetriken}
    \section{Unit-Tests}

    \subsection{Anzahl Unit-Tests}
    In dieser Tabelle werden die Anzahl der Unit-Tests vor und nach der Qualitätssicherung verglichen. Wir versuchten, unsere Tests aus der Implementierung
    auszubauen, um weitere Fehler zu finden und die Überdeckung zu erhöhen. Das Paket...

    \vspace{20pt}
    \begin{table}[h]
        \begin{center}
        \begin{tabular}{ | l | c | c | }
            \hline
            Paket & Unit-Tests vor & Unit-Tests nach  \\
            & Qualitätssicherung & Qualitätssicherung \\ \hline
            User & 16 & 16 \\
            Database & 44 & 44 \\
            Game & 45 & 45 \\
            Gamification & 34 & 34 \\
            Configuration & 10 & 10 \\
            \hline
        \end{tabular}
        \end{center}
        \caption{Vergleich der Anzahl an Unit-Tests}
    \end{table}%

    \subsection{Zeilenüberdeckung}
    Die Tabelle zeigt die verschiedenen Zeilenüberdeckungswerte vor und nach der Qualitätssicherung. Man erkennt..

    \vspace{20pt}
    \begin{table}[h]
        \begin{center}
        \begin{tabular}{ | l | c | c | }
            \hline
            Paket & Zeilenüberdeckung vor  & Zeilenüberdeckung nach \\
            & Qualitätssicherung (\%)  & Qualitätssicherung (\%) \\ \hline
            User & 60 & 16 \\
            Database & 62 & 44 \\
            Game & 74 & 45 \\
            Gamification & 96 & 34 \\
            Configuration & 76 & 10 \\
            \hline
        \end{tabular}
        \end{center}
        \caption{Vergleich der Zeilenüberdeckung}
    \end{table}%

    \newpage

    \subsection{Methodenüberdeckung}
    Bei der Methodenüberdeckung..

    \vspace{20pt}
    \begin{table}[h]
        \begin{center}
        \begin{tabular}{ | l | c | }
            \hline
            Paket & Methodenüberdeckung nach \\
            & Qualitätssicherung (\%) \\ \hline
            User & 60 \\
            Database & 62 \\
            Game & 74 \\
            Gamification & 96 \\
            Configuration & 76  \\
            \hline
        \end{tabular}
        \end{center}
        \caption{Methodenüberdeckung}
    \end{table}%

    \subsection{Klassenüberdeckung}
    Die Klassenüberdeckung sollte maximal sein, denn..

    \vspace{20pt}
    \begin{table}[h]
        \begin{center}
        \begin{tabular}{ | l | c | }
            \hline
            Paket & Klassenüberdeckung nach \\
            & Qualitätssicherung (\%) \\ \hline
            User & 60 \\
            Database & 62 \\
            Game & 74 \\
            Gamification & 96 \\
            Configuration & 76  \\
            \hline
        \end{tabular}
        \end{center}
        \caption{Klassenüberdeckung}
    \end{table}%


    \chapter{Fehlerbehebungen}

    \section{GitHub Issues}
    Im Folgenden sind die wichtigsten Fehler mit Fehlersympton, Grund und deren Behebung aufgelistet. Sie beziehen sich alle auf Issues
    aus unserem GitHub-Repository.

    \subsection{Spielladefehler (Issue 16)}
    \begin{itemize}
        \item Fehlersymptom: Organisator verlor seine Liste an Spielen nach einem Server-Neustart
        \item Grund: Spiele wurden in einer Map gecached, welche beim Neustart nicht neu geladen wurde
        \item Behebung: Spiele werden beim Neustart aus der Datenbank geladen
    \end{itemize}

    \subsection{Spielduplizierung (Issue 18)}
    \begin{itemize}
        \item Fehlersymptom: Spiele, zu denen ein Spieler eingeladen war, wurden doppelt angezeigt
        \item Grund: Spieleinladungen wurden auch von getActiveGames zurückgegeben
        \item Behebung: Fehler in der Methode getActiveGames behoben
    \end{itemize}

    \subsection{Kein Start von valider Runde (Issue 19)}
    \begin{itemize}
        \item Fehlersymptom: Wenn startRound mit einer validen Spiel-ID aufgerufen wurde, wurde null zurückgegeben
        \item Grund: Bei der Spielerstellung wurde der FeatureSet-Name nicht übermittelt sowie die isTerminated-Spalte des Spiels
        in der Datenbank nicht gesetzt, was zu einem Null-Pointer beim Ausführen von startRound führte
        \item Behebung: Die fehlende Übermittlung des FeatureSet-Namen wurde hinzugefügt und die isTerminated-Spalte korrekt gesetzt
    \end{itemize}

    \subsection{Mehrfachmarkierung (Issue 20)}
    \begin{itemize}
        \item Fehlersymptom: Merkmale können als unwichtig markiert und für die Runde ausgewählt werden
        \item Grund: Es wurde nicht überprüft, ob ein Merkmal als unwichtig markiert wurde, bevor es ausgewählt werden kann
        \item Behebung: Überprüfen, ob ein Merkmal unwichtig ist, bevor es markiert wird/werden kann
    \end{itemize}

    \subsection{Graphen für nutzlos markierte Merkmale (Issue 21)}
    \begin{itemize}
        \item Fehlersymptom: Wenn für ein unwichtiges Merkmal die Graphen angezeigt werden, kann man nicht mehr mit dem Spiel interagieren
        \item Grund: Die ganze Karte, in der eine Feature-Box angezeigt wird, wird ausgegraut, womit das Betätigen des Schließen-Knopfs des Modals nicht mehr klickbar ist
        \item Behebung: Verhindern, dass Graphen für unwichtige Features angezeigt werden können
    \end{itemize}

    \subsection{Masterpasswort gleich Nutzername (Issue 30)}
    \begin{itemize}
        \item Fehlersymptom: Bei der Registrierung wird der Nutzername als Masterpasswort eingetragen, wenn von Spieler auf Organisator gewechselt wird
        \item Grund: Beide Werte werden in der selben Variable gespeichert
        \item Behebung: Variable löschen beim Wechsel
    \end{itemize}

    \subsection{Einladungen bleiben erhalten (Issue 32)}
    \begin{itemize}
        \item Fehlersymptom: Wenn Einladungen vom Spieler angenommen werden, werden sie nicht von der Einladungsliste im Front-End entfernt
        \item Grund: War nicht implementiert
        \item Behebung: Einladungen von der Einladungsliste löschen und auf die Spiele-Liste setzen
    \end{itemize}

    \subsection{Umlaute (Issue 37, 160, 170)}
    \begin{itemize}
        \item Fehlersymptom: Umlaute werden im Front-End nicht richtig dargestellt
        \item Grund: HTML-Tag für UTF-8 nicht geschlossen und ResourceBundle-Dateien können kein UTF-8
        \item Behebung: Tag schließen und Oracle-Unicode-Konverter auf die Localisation-Dateien anwenden
    \end{itemize}

    \subsection{Überspringen setzt Streak nicht zurück (Issue 56)}
    \begin{itemize}
        \item Fehlersymptom: Überspringen einer Runde setzt die Streak eines Spielers nicht zurück
        \item Grund: API hat eine Exception geworfen, bevor sie zum Überspringen-Aufruf gekommen ist
        \item Behebung: Exception behoben
    \end{itemize}

    \subsection{Zeilenumbruch Darstellung (Issue 61)}
    \begin{itemize}
    \item Fehlersymptom: Zeilenumbrüche in Hilfetexten und Merkmalstexten werden als \textbackslash n dargestellt
    \item Grund: HTML versteht \textbackslash n nicht als Zeilenumbruch
    \item Behebung: Im Front-End alle \textbackslash n durch <br> ersetzen
    \end{itemize}

    \subsection{Runden mit selben Merkmalen in anderer Reihenfolge (Issue 79)}
    \begin{itemize}
    \item Fehlersymptom: Runden mit den selben Merkmalen in einer anderen
     Reihenfolge wurden nicht ausgeschlossen
    \item Grund: Nicht implementiert, nur die exakte Abfolge der Feature-IDs wurde
     überprüft
    \item Behebung: Feature-IDs werden beim Schreiben in die Datenbank sortiert
    \end{itemize}

    \subsection{Server stürzt ohne Config-Datei ab (Issue 81)}
    \begin{itemize}
    \item Fehlersymptom: Wenn die Datei config.properties nicht wie erwartet
     existiert, konnte man den Server trotzdem starten und es gab keine
     adäquate Fehlermeldung
    \item Grund: Nicht implementiert
    \item Behebung: Exception wird geworfen, wenn Fehler beim Lesen der Config
     auftreten, diese wird im Frontend behandelt
    \end{itemize}

    \subsection{Nutzlose Merkmale nicht gespeichert (Issue 89)}
    \begin{itemize}
    \item Fehlersymptom: In Runden, die übersprungen wurden, wurden als nutzlos
     markierte Merkmale nicht gespeichert
    \item Grund: Fehler beim Parsen aus den Front-End-Informationen
    \item Behebung: Parsen korrigiert
    \end{itemize}

    \subsection{Keine Merkmale angezeigt (Issue 99)}
    \begin{itemize}
    \item Fehlersymptom: Wenn in Matrix-Select eine Merkmalsanzahl angegeben wurde,
    die keine Quadratzahl war, wurden die Features nicht angezeigt
    \item Grund: Front-End konnte nur mit Quadratzahlen umgehen
    \item Behebung: Flexibleres Design für die Merkmalsdarstellung, das auch mit anderen
     Zahlen umgehen kann
    \end{itemize}

    \subsection{Übersprungene Runden zählen für Spielterminierung (Issue 137)}
    \begin{itemize}
    \item Fehlersymptom: Übersprungene Runden haben als normale Runden für die
     Runden-Terminierung gezählt, ein Spiel konnte also auch terminieren, wenn alle
      Runden übersprungen wurden
    \item Grund: Keine Unterscheidung zwischen normal beendeten und übersprungenen
     Runden 
    \item Behebung: Neues Skipped-Attribut für Runden, welches angibt, ob eine
     Runde übersprungen wurde und zum Filtern benutzt wird
    \end{itemize}

    \subsection{Runden mit weniger Features nicht gespeichert (Issue 153)}
    \begin{itemize}
    \item Fehlersymptom: Runden, bei denen weniger Features als normal ausgewählt
     wurden, wurden nicht gespeichert und es wurden -1 Punkte zurückgegeben
    \item Grund: Ursprünglich nicht erwartetes Verhalten, dass weniger Features
     als erwartet ausgewählt werden, hat Ausnahmebehandlung ausgelöst
    \item Behebung: Ausnahmebehandlung entfernt
    \end{itemize}

    \subsection{Nächste Runde, ohne zuvor Merkmale auszuwählen (Issue 164)}
    \begin{itemize}
    \item Fehlersymptom: Wenn Merkmale ausgewählt wurden und dann alle Merkmale
     als nutzlos markiert wurden, konnte trotzdem der Bestätigen-Knopf betätigt
     werden und das Spiel ist eingefroren
    \item Grund: Aufruf an den ML-Server ohne ausgewählte Features und ohne
     Ausnahmebehandlung für diesen Spezialfall
    \item Behebung: Knopf kann nicht mehr in diesem Szenario betätigt werden, Ausnahmebehandlung
    \end{itemize}

    \subsection{Graph anzeigen nach Beendigung der Runde (Issue 197)}
    \begin{itemize}
        \item Fehlersymptom: Wenn die Runde bestätigt wurde und direkt danach ein Graph anzeigt wurde, ist das Spiel eingefroren
        \item Grund: Die neue Runde wurde noch nicht geladen, aber der Graph eines Merkmals der letzten Runde wurde noch angezeigt
        \item Behebung: Graphen können erst wieder bei den Merkmalen in der neuen Runde angezeigt werden
    \end{itemize}

\end{document}
