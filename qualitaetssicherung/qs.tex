\documentclass[a4paper]{scrreprt}

\usepackage[ngerman]{babel}
\usepackage[utf8]{inputenc}
\usepackage[T1]{fontenc}
\usepackage{ae}
\usepackage[bookmarks, bookmarksnumbered]{hyperref}
\usepackage{tabularx}
\usepackage{graphicx}
\usepackage{csquotes}
\usepackage{verbatim}
\usepackage{float}
\usepackage[nonumberlist, toc, section]{glossaries}
\usepackage[german]{fancyref}

\setcounter{secnumdepth}{4}

\begin{document}

    \title{Qualitätssicherungsdokument CS:Select}
    \author{Luca Springer, Alexander Linder, Julian Dinh, Nicholas Bieker,\\ Bendix Sonnenberg}
    \maketitle

    \tableofcontents
    \chapter{Einleitung}



    \chapter{Testfälle}
        \section{Pflichtenheftszenarien}
        Bei den folgenden Szenarien läuft das System bereits (entweder auf localhost oder auf dem KIT-Server). \\
            \subsection{Spieler registrieren}
            \subsubsection{Spieler mit gültiger Email registrieren}
            \begin{itemize}
                \item Schritte:
                    \begin{itemize}
                        \item Verbinden mit localhost:8080/CS-Select
                        \item Auf "Brauchst du einen Account?" klicken
                        \item Gültige Email-Adresse in das Email-Feld einsetzen ("player@csselect.com")
                        \item Beliebiges Passwort in Passwort-Feld eingeben ("testPassword")
                        \item Passwort wiederholen
                        \item "Player" auswählen
                        \item Auf "Registrieren" klicken
                    \end{itemize}
                \item OK %%oder Fehler
            \end{itemize}



    \chapter{Testüberdeckung}
      \section{Unit-Tests}




    \chapter{Testmetriken}




    \chapter{Fehlerbehebungen}

    \begin{itemize}
        \item Fehlersympton:
        \item Grund:
        \item Behebung:
    \end{itemize}



\end{document}