\documentclass[a4paper]{scrreprt}

\usepackage[ngerman]{babel}
\usepackage[utf8]{inputenc}
\usepackage[T1]{fontenc}
\usepackage{ae}
\usepackage[bookmarks, bookmarksnumbered]{hyperref}
\usepackage{tabularx}
\usepackage{graphicx}
\usepackage{csquotes}
\usepackage{verbatim}
\usepackage{float}
\usepackage[nonumberlist, toc, section]{glossaries}
\usepackage[german]{fancyref}

\setcounter{secnumdepth}{4}

\begin{document}

    \title{Qualitätssicherungsdokument CS:Select}
    \author{Luca Springer, Alexander Linder, Julian Dinh, Nicholas Bieker,\\ Bendix Sonnenberg}
    \maketitle

    \tableofcontents
    \chapter{Einleitung}
    Die Qualitätssicherung ist die abschließende Phase unseres Projekt CS:Select. In dieser Phase wurde das Produkt
    gründlich und umfangreich getestet. Jegliche Fehler, welche nach der Implementierungsphase entstanden sind, sollten einerseits
    gefunden und andererseits beseitigt werden. Außerdem ging es darum,
    die Code-Qualität noch weiter zu verbessern und auch schon vorhandene Unit-Tests weiter auszubauen. \\
    \\
    In diesem Dokument befassen wir uns insbesondere damit, wie wir mit dem Testen vorgegangen sind. Als Grundlage dazu
    dienen die Testfallszenarien aus dem Pflichtenheft, welche wir abgearbeitet haben, aber auch eigene Testfälle, welche wir hinzugefügt haben. Außerdem
    haben auch andere Personen, welche nicht an dem Produkt gearbeitet haben, das Produkt ausprobiert und uns Feedback gegeben. Daraufhin
    konnten wir das Produkt noch weiter verbessern, z.B. wenn die Funktionsweise an manchen Stellen nicht intuitiv genug für den Nutzer war. \\
    \\
    Als weiteren Punkt beschreiben wir unsere Testüberdeckungswerte und wie diese erreicht wurden. Darüber hinaus... \\ %TODO
    \\
    Schließlich sind durch all diese Testverfahren diverse Fehler erkannt und beseitigt worden. Im letzten Teil des Dokuments
    reflektieren wir noch einmal diese aufgetretenen Fehler. Dabei geht es auch daraum, warum sie überhaupt auftraten
    und wie sie von uns behoben worden sind.





    \chapter{Testfälle}
        \section{Pflichtenheftszenarien}
        Die folgenden Szenarien sind alle dem Pflichtenheft entnommen. \\
            \subsection{Spieler registrieren}
            \begin{itemize}
                \item Ausgangssituation: Spieler befindet sich auf der Login-Seite
                \item Schritte:
                    \begin{itemize}
                        \item Auf \enquote{Brauchst du einen Account?} klicken
                        \item Gültige E-Mail-Adresse in das E-Mail-Feld eingeben (\enquote{spieler@csselect.com})
                        \item Beliebiges Passwort in Passwort-Feld eingeben (\enquote{testPasswort})
                        \item Passwort wiederholen (\enquote{testPasswort})
                        \item \enquote{Spieler} auswählen
                        \item Einen Nutzernamen eingeben (\enquote{spieler})
                        \item Auf \enquote{Registrieren} klicken
                    \end{itemize}
                \item Funktioniert: Ja
            \end{itemize}

            \subsection{Spieler anmelden}
            \begin{itemize}
                \item Ausgangssituation: Spieler befindet sich auf der Login-Seite und besitzt bereits einen Account
                \item Schritte:
                    \begin{itemize}
                        \item Registrierte E-Mail-Adresse in das E-Mail-Feld eingeben (\enquote{spieler@csselect.com})
                        \item Dazugehöriges Passwort in Passwort-Feld eingeben (\enquote{testPasswort})
                        \item Auf \enquote{Anmelden} klicken
                    \end{itemize}
                \item Funktioniert: Ja
                \item Änderungen zum Pflichtenheftszenario
                \begin{itemize}
                    \item Es gibt keine Checkbox für den Spieler, man muss die Organisator-Checkbox leer lassen
                \end{itemize}
            \end{itemize}

            \subsection{Spieler abmelden}
            \begin{itemize}
                \item Ausgangssituation: Spieler ist angemeldet
                \item Schritte:
                    \begin{itemize}
                        \item In der Navigationsleiste auf \enquote{Abmelden} klicken
                    \end{itemize}
                \item Funktioniert: Ja
                \item Änderungen zum Pflichtenheftszenario
                    \begin{itemize}
                        \item Vorgang muss nicht bestätigt werden
                    \end{itemize}
            \end{itemize}

            \subsection{Organisator registrieren}
            \begin{itemize}
                \item Ausgangssituation: Organisator befindet sich auf der Login-Seite
                \item Schritte:
                    \begin{itemize}
                        \item Auf \enquote{Brauchst du einen Account?} klicken
                        \item Gültige E-Mail-Adresse in das E-Mail-Feld eingeben (\enquote{organiser@csselect.com})
                        \item Beliebiges Passwort in Passwort-Feld eingeben (\enquote{testPasswort})
                        \item Passwort wiederholen (\enquote{testPasswort})
                        \item \enquote{Organisator} auswählen
                        \item Das Master-Passwort eingeben (\enquote{sicherespasswort})
                        \item Auf \enquote{Registrieren} klicken
                    \end{itemize}
            \item Funktioniert: Ja
            \end{itemize}

            \subsection{Organisator anmelden}
            \begin{itemize}
                \item Ausgangssituation: Organisator befindet sich auf der Login-Seite und besitzt bereits einen Account
                \item Schritte:
                    \begin{itemize}
                        \item Registrierte E-Mail-Adresse in das E-Mail-Feld eingeben (\enquote{spieler@csselect.com})
                        \item Dazugehöriges Passwort in Passwort-Feld eingeben (\enquote{testPasswort})
                        \item \enquote{Organisator} auswählen
                        \item Auf \enquote{Anmelden} klicken
                    \end{itemize}
                \item Funktioniert: Ja
            \end{itemize}

            \subsection{Organisator abmelden}
            \begin{itemize}
                \item Ausgangssituation: Organisator ist angemeldet
                \item Schritte:
                    \begin{itemize}
                        \item In der Navigationsleiste auf \enquote{Abmelden} klicken
                    \end{itemize}
                \item Funktioniert: Ja
                \item Änderungen zum Pflichtenheftszenario
                    \begin{itemize}
                        \item Vorgang muss nicht bestätigt werden
                    \end{itemize}
            \end{itemize}

            \subsection{Organisator löschen}

            \subsection{Spiel erstellen}

            \subsection{Spieler nach Spielbeginn einladen}

            \subsection{Spiel vorzeitig beenden}

            \subsection{Spiel löschen}

            \subsection{Spieleinstellungen speichern}

            \subsection{Spieleinstellungen laden}

            \subsection{Spielzustand auslesen}

            \subsection{Einladung zu Spiel annehmen}

            \subsection{Runde spielen: Matrix Select}

            \subsection{Runde spielen: Binär Select}

            \subsection{Spielterminierung während Runde}
            % Das machen wir ja nicht?

            \subsection{Auswahl überspringen}

            \subsection{Merkmal als unwichtig markieren}

            \subsection{Streak erreichen}

            \subsection{Runde starten}

            \subsection{Achievements ansehen}

            \subsection{Leaderboard ansehen}

            \subsection{Sprache ändern}

            \subsection{Konfiguration bearbeiten}

            \subsection{Server beenden}

            \subsection{Passwort ändern}

            \subsection{E-Mail ändern}

            \subsection{Passwort zurücksetzen}

            \subsection{Hilfe-Schaltfläche verwenden}


        \section{Weitere Testfallszenarien}
            \subsection{Nutzer bei Inaktivität abmelden}


    \chapter{Hallway Usability Testing}



    \chapter{Testüberdeckung}
      \section{Unit-Tests}



    \chapter{Testmetriken}



    \chapter{Fehlerbehebungen}

    \begin{itemize}
        \item Fehlersymptom:
        \item Grund:
        \item Behebung:
    \end{itemize}

\end{document}