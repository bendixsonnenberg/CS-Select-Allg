\documentclass[a4paper]{scrreprt}

\usepackage[ngerman]{babel}
\usepackage[utf8]{inputenc}
\usepackage[T1]{fontenc}
\usepackage{ae}
\usepackage[bookmarks, bookmarksnumbered]{hyperref}
\usepackage{tabularx}
\usepackage{graphicx}
\usepackage{csquotes}
\usepackage{verbatim}
\usepackage{float}
\usepackage[nonumberlist, toc, section]{glossaries}
\usepackage[german]{fancyref}

\setcounter{secnumdepth}{4}

\begin{document}

    \title{Qualitätssicherungsdokument CS:Select}
    \author{Luca Springer, Alexander Linder, Julian Dinh, Nicholas Bieker,\\ Bendix Sonnenberg}
    \maketitle

    \tableofcontents
    \chapter{Einleitung}
    Die Qualitätssicherung ist die abschließende Phase unseres Projekt CS:Select. In dieser Phase wurde das Produkt
    gründlich und umfangreich getestet. Jegliche Fehler, welche nach der Implementierungsphase entstanden sind, sollten einerseits
    gefunden und andererseits beseitigt werden. Außerdem ging es darum,
    die Code-Qualität noch weiter zu verbessern und auch schon vorhandene Unit-Tests weiter auszubauen. \\
    \\
    In diesem Dokument befassen wir uns insbesondere damit, wie wir mit dem Testen vorgegangen sind. Als Grundlage dazu
    dienen die Testfallszenarien aus dem Pflichtenheft, welche wir abgearbeitet haben, aber auch eigene Testfälle, welche wir hinzugefügt haben. Außerdem
    haben auch andere Personen, welche nicht an dem Produkt gearbeitet haben, das Produkt ausprobiert und uns Feedback gegeben. Daraufhin
    konnten wir das Produkt noch weiter verbessern, z.B. wenn die Funktionsweise an manchen Stellen nicht intuitiv genug für den Nutzer war. \\
    \\
    Als weiteren Punkt beschreiben wir unsere Testüberdeckungswerte und wie diese erreicht wurden. Darüber hinaus... \\ %TODO
    \\
    Schließlich sind durch all diese Testverfahren diverse Fehler erkannt und beseitigt worden. Im letzten Teil des Dokuments
    reflektieren wir noch einmal diese aufgetretenen Fehler. Dabei geht es auch daraum, warum sie überhaupt auftraten
    und wie sie von uns behoben worden sind.





    \chapter{Testfälle}

        \section{Pflichtenheftszenarien}
        Die folgenden Szenarien sind alle dem Pflichtenheft entnommen. Sie sind mit dem deutschen Sprachpaket durchgeführt worden.

            \subsection{Spieler registrieren}
            \begin{itemize}
                \item Ausgangssituation: Spieler befindet sich auf der Login-Seite
                \item Schritte:
                    \begin{itemize}
                        \item Auf \enquote{Brauchst du einen Account?} klicken
                        \item Gültige E-Mail-Adresse in das E-Mail-Feld eingeben (\enquote{spieler@csselect.com})
                        \item Beliebiges Passwort in Passwort-Feld eingeben (\enquote{testPasswort})
                        \item Passwort wiederholen (\enquote{testPasswort})
                        \item \enquote{Spieler} auswählen
                        \item Einen Nutzernamen eingeben (\enquote{spieler})
                        \item Auf \enquote{Registrieren} klicken
                    \end{itemize}
                \item Funktioniert: Ja
            \end{itemize}

            \subsection{Spieler anmelden}
            \begin{itemize}
                \item Ausgangssituation: Spieler befindet sich auf der Login-Seite und besitzt bereits einen Account
                \item Schritte:
                    \begin{itemize}
                        \item Registrierte E-Mail-Adresse in das E-Mail-Feld eingeben (\enquote{spieler@csselect.com})
                        \item Dazugehöriges Passwort in Passwort-Feld eingeben (\enquote{testPasswort})
                        \item Auf \enquote{Anmelden} klicken
                    \end{itemize}
                \item Funktioniert: Ja
                \item Änderungen zum Pflichtenheftszenario
                \begin{itemize}
                    \item Es gibt keine Checkbox für den Spieler, man muss die Organisator-Checkbox leer lassen
                \end{itemize}
            \end{itemize}

            \subsection{Spieler abmelden}
            \begin{itemize}
                \item Ausgangssituation: Spieler ist angemeldet
                \item Schritte:
                    \begin{itemize}
                        \item In der Navigationsleiste auf \enquote{Abmelden} klicken
                    \end{itemize}
                \item Funktioniert: Ja
                \item Änderungen zum Pflichtenheftszenario
                    \begin{itemize}
                        \item Vorgang muss nicht bestätigt werden
                    \end{itemize}
            \end{itemize}

            \subsection{Organisator registrieren}
            \begin{itemize}
                \item Ausgangssituation: Organisator befindet sich auf der Login-Seite
                \item Schritte:
                    \begin{itemize}
                        \item Auf \enquote{Brauchst du einen Account?} klicken
                        \item Gültige E-Mail-Adresse in das E-Mail-Feld eingeben (\enquote{organiser@csselect.com})
                        \item Beliebiges Passwort in Passwort-Feld eingeben (\enquote{testPasswort})
                        \item Passwort wiederholen (\enquote{testPasswort})
                        \item \enquote{Organisator} auswählen
                        \item Das Master-Passwort eingeben (\enquote{sicherespasswort})
                        \item Auf \enquote{Registrieren} klicken
                    \end{itemize}
            \item Funktioniert: Ja
            \end{itemize}

            \subsection{Organisator anmelden}
            \begin{itemize}
                \item Ausgangssituation: Organisator befindet sich auf der Login-Seite und besitzt bereits einen Account
                \item Schritte:
                    \begin{itemize}
                        \item Registrierte E-Mail-Adresse in das E-Mail-Feld eingeben (\enquote{spieler@csselect.com})
                        \item Dazugehöriges Passwort in Passwort-Feld eingeben (\enquote{testPasswort})
                        \item \enquote{Organisator} auswählen
                        \item Auf \enquote{Anmelden} klicken
                    \end{itemize}
                \item Funktioniert: Ja
            \end{itemize}

            \subsection{Organisator abmelden}
            \begin{itemize}
                \item Ausgangssituation: Organisator ist angemeldet
                \item Schritte:
                    \begin{itemize}
                        \item In der Navigationsleiste auf \enquote{Abmelden} klicken
                    \end{itemize}
                \item Funktioniert: Ja
                \item Änderungen zum Pflichtenheftszenario
                    \begin{itemize}
                        \item Vorgang muss nicht bestätigt werden
                    \end{itemize}
            \end{itemize}

            \subsection{Organisator löschen}

            \subsection{Spiel erstellen}

            \subsection{Spieler nach Spielbeginn einladen}

            \subsection{Spiel vorzeitig beenden}

            \subsection{Spiel löschen}

            \subsection{Spieleinstellungen speichern}

            \subsection{Spieleinstellungen laden}

            \subsection{Spielzustand auslesen}

            \subsection{Einladung zu Spiel annehmen}
            \begin{itemize}
                \item Ausgangssituation: Spieler ist angemeldet und wurde zu einem Spiel eingeladen
                \item Schritte:
                    \begin{itemize}
                        \item Auf der Übersichts-Seite das gewünschte Spiel aussuchen und auf \enquote{Akzeptieren} klicken
                    \end{itemize}
                \item Funktioniert: Ja
            \end{itemize}
            % TODO: Option 2

            \subsection{Runde spielen: Matrix Select}

            \subsection{Runde spielen: Binär Select}

            \subsection{Spielterminierung während Runde}
            % Das machen wir ja nicht?

            \subsection{Auswahl überspringen}
            \begin{itemize}
                \item Ausgangssituation: Spieler befindet sich auf der Spiel-Seite
                \item Schritte:
                    \begin{itemize}
                        \item Gegebenenfalls Merkmale als nutzlos markieren
                        \item Auf \enquote{Überspringen} klicken
                    \end{itemize}
                \item Funktioniert: Ja
                \item Änderungen zum Pflichtenheftszenario
                \begin{itemize}
                    \item Spieler verliert keine Punkte, bekommt aber auch keine
                    \item Streak wird zurückgesetzt, es sei denn Spieler markiert alle Merkmale als nutzlos
                \end{itemize}
            \end{itemize}

            \subsection{Merkmal als unwichtig markieren}
            \begin{itemize}
                \item Ausgangssituation: Spieler befindet sich auf der Spiel-Seite
                \item Schritte:
                    \begin{itemize}
                        \item Bei einem oder mehreren unwichtigen Merkmalen auf \enquote{Nutzlos markieren} klicken
                        \item Zusätzlich kann Spieler auch noch Merkmale für seine Merkmalsauswahl aussuchen
                    \end{itemize}
                \item Funktioniert: Ja
            \end{itemize}

            \subsection{Streak erreichen}
            \begin{itemize}
                \item Ausgangssituation: Spieler befindet sich auf der Spiel-Seite
                \item Schritte:
                    \begin{itemize}
                        \item Mehrere Runden (mind. eine) am Stück spielen
                        \item Im Feld \enquote{Streak} anhand des Balkens und des zugehörigen Texts den Zustand seiner Streak erkennen
                    \end{itemize}
                    \item Funktioniert: Ja
            \end{itemize}

            \subsection{Runde starten}
            \begin{itemize}
                \item Ausgangssituation: Spieler ist angemeldet und hat mindestens ein aktives Spiel
                \item Schritte:
                    \begin{itemize}
                        \item Auf der Übersichts-Seite das gewünschte Spiel aussuchen und auf \enquote{Spielen} klicken
                    \end{itemize}
                \item Funktioniert: Ja
            \end{itemize}

            \subsection{Achievements ansehen}
            \begin{itemize}
                \item Ausgangssituation: Spieler ist angemeldet
                \item Schritte:
                    \begin{itemize}
                        \item In der Navigationsleiste auf \enquote{Achievements} klicken
                    \end{itemize}
                \item Funktioniert: Ja
            \end{itemize}

            \subsection{Leaderboard ansehen}
            \begin{itemize}
                \item Ausgangssituation: Spieler ist angemeldet
                \item Schritte:
                    \begin{itemize}
                        \item Direkt auf der Übersichts-Seite die Top 5 einsehen
                    \end{itemize}
                \item Funktioniert: Ja
            \end{itemize}

            \subsection{Sprache ändern}
            \begin{itemize}
                \item Ausgangssituation: Nutzer ist angemeldet
                \item Schritte:
                    \begin{itemize}
                        \item In der Navigationsleiste auf \enquote{Einstellungen} klicken
                        \item Auf Feld \enquote{Sprache} klicken und anschließend eine der verfügbaren Sprachen auswählen
                        \item Auf \enquote{Aktualisieren} klicken
                    \end{itemize}
                \item Funktioniert: Ja
                \item Anmerkung: Seite muss neu geladen werden, damit die Änderung in Kraft tritt
            \end{itemize}

            \subsection{Konfiguration bearbeiten}


            \subsection{Server beenden}


            \subsection{Passwort ändern}
            \begin{itemize}
                \item Ausgangssituation: Nutzer ist angemeldet
                \item Schritte:
                    \begin{itemize}
                        \item In der Navigationsleiste auf \enquote{Einstellungen} klicken
                        \item In das Textfeld \enquote{Neues Passwort} das neue Passwort schreiben
                        \item Neues Passwort im Textfeld \enquote{Passwort wiederholen} wiederholen
                        \item Auf \enquote{Aktualisieren} klicken
                    \end{itemize}
                \item Funktioniert: Ja
                \item Änderungen zum Pflichtenheftszenario
                \begin{itemize}
                    \item Das aktuelle Passwort muss nicht angegeben werden
                \end{itemize}
            \end{itemize}

            \subsection{E-Mail ändern}
            \begin{itemize}
                \item Ausgangssituation: Nutzer ist angemeldet
                \item Schritte:
                    \begin{itemize}
                        \item In der Navigationsleiste auf \enquote{Einstellungen} klicken
                        \item In das Textfeld \enquote{Neue E-Mail} eine neue gültige E-Mail einfügen
                        \item Auf \enquote{Aktualisieren} klicken
                    \end{itemize}
                \item Funktioniert: Ja
                \item Änderungen zum Pflichtenheftszenario
                    \begin{itemize}
                        \item Das aktuelle Passwort muss nicht angegeben werden
                    \end{itemize}
            \end{itemize}

            \subsection{Passwort zurücksetzen}
            \begin{itemize}
                \item Ausgangssituation: Nutzer befindet sich auf der Login-Seite
                \item Schritte:
                    \begin{itemize}
                        \item Auf \enquote{Du hast dein Passwort vergessen?} klicken
                        \item Seine E-Mail in das E-Mail-Feld einfügen
                        \item Falls man Organisator ist: \enquote{Organisator} auswählen
                        \item Auf \enquote{Passwort zurücksetzen} klicken
                        \item Mit dem per E-Mail erhaltenen temporären Passwort anmelden
                    \end{itemize}
                \item Funktioniert: Ja
            \end{itemize}

            \subsection{Hilfe-Schaltfläche verwenden}
            % Ist anders als im Pflichtenheft



        \section{Weitere Testfallszenarien}
            \subsection{Nutzer bei Inaktivität abmelden}
            \begin{itemize}
                \item Ausgangssituation: Nutzer bist angemeldet
                \item Schritte:
                    \begin{itemize}
                        \item Für zehn Minuten nichts im Browser-Fenster tun
                    \end{itemize}
                \item Funktioniert: Ja
            \end{itemize}


    \chapter{Hallway Usability Testing}



    \chapter{Testüberdeckung}
        \section{Unit-Tests}
                \begin{table}[h]
                    \centering
                \begin{tabular}{ | l | c | c | }
                    \hline
                    Paket & Testüberdeckung (\%) & Unit-Tests \\ \hline
                    User & 60 & 16 \\
                    Database & 62 & 44 \\
                    Game & 74 & 45 \\
                    Gamification & 96 & 34 \\
                    Configuration & 76 & 10 \\
                    \hline
                \end{tabular}
                \caption{Testüberdeckung vor Qualitätssicherung}
                \end{table}%

                \begin{table}[h]
                    \centering
                \begin{tabular}{ | l | c | c | }
                    \hline
                    Paket & Testüberdeckung (\%) & Unit-Tests \\ \hline
                    User & 60 & 16 \\
                    Database & 62 & 44 \\
                    Game & 74 & 45 \\
                    Gamification & 96 & 34 \\
                    Configuration & 76 & 10 \\
                    \hline
                \end{tabular}
                \caption{Testüberdeckung nach Qualitätssicherung}
                \end{table}%

    Die beiden Tabellen zeigen die Testüberdeckungswerte vor und nach der Qualitätssicherung. Man erkennt..



    \chapter{Testmetriken}



    \chapter{Fehlerbehebungen}

    \subsection{Spielladefehler (Issue 16)}
    \begin{itemize}
        \item Fehlersymptom: Ein Organisator verlor seine Liste an Spielen nach einem Server-Neustart
        \item Grund: Spiele wurden in einer Map gecached, welche beim Neustart nicht neu geladen wurde
        \item Behebung: Spiele werden beim Neustart aus der Datenbank geladen
    \end{itemize}
    \subsection{Spielduplizierung (Issue 18)}
    \begin{itemize}
        \item Fehlersymptom: Spiele zu denen ein Spieler eingeladen war wurden doppelt angezeigt
        \item Grund: Spieleinladungen wurden auch von getActiveGames zurückgegeben
        \item Behebung: Fehler in der Methode getActiveGames behoben
    \end{itemize}
    \subsection{startRound null (Issue 19)}
    \begin{itemize}
        \item Fehlersymptom: Wenn startRound mit einer validen Spiel-ID aufgerufen wurde wurde null zurückgegeben
        \item Grund: Bei der Spielerstellung wurde der FeatureSet-Name nicht übermittelt, sowie die "isTerminated"-Spalte des Spiels
        in der Datenbank nicht gesetzt, was zu einem Nullpointer beim ausführen von startRound führte
        \item Behebung: Die fehlende Übermittlung des FeatureSet-Namen wurde hinzugefügt, sowie die "isTerminated"-Spalte korrekt gesetzt
    \end{itemize}
    \subsection{Mehrfachmarkierung (Issue 20)}
    \begin{itemize}
        \item Fehlersymptom: Merkmale können als unwichtig markiert werden und für die Runde ausgewählt werden
        \item Grund: Es wurde nicht überprüft ob ein Merkmal als unwichtig markiert wurde bevor es ausgewählt werden kann
        \item Behebung: Überprüfen ob ein Merkmal unwichtig ist, bevor es markiert wird/werden kann
    \end{itemize}
    \subsection{Einfrieren des Spiels bei anzeigen der Graphen für unwichtige Features(Issue 21)}
    \begin{itemize}
        \item Fehlersymptom: Wenn für ein unwichtiges Merkmal die Graphen angezeigt werden, kann man nicht mehr mit dem Spiel interagieren.
        \item Grund: Die ganze Karte in der ein Feature-Box angezeigt wird, wird ausgegraut womit das Betätigen des Schließen Knopf des Modals nicht mehr klickbar ist.
        \item Behebung: Verhindern das Graphen für unwichtige Features angezeigt werden können
    \end{itemize}
    \subsection{Linebreaks Darstellung(Issue 61)}
    \begin{itemize}
        \item Fehlersymptom: Linebreaks in Hilfetexten und Merkmalstexten werden als \textbackslash n dargestellt
        \item Grund: HTML versteht \textbackslash n nicht als Linebreak
        \item Behebung: Im Frontend alle \textbackslash n durch <br> ersetzten
    \end{itemize}
    \subsection{Masterpasswort gleich Nutzername ?(Issue 30)}
    \begin{itemize}
        \item Fehlersymptom: Bei der Registrierung wird der Nutzername als Masterpasswort eingetragen, wenn von Spieler auf Organisator gewechselt wird.
        \item Grund: Bei Werte werden in der selben Variable gespeichert
        \item Behebung: Variable löschen beim Wechsel
    \end{itemize}
    \subsection{Einladungen bleiben erhalten(Issue 32)}
    \begin{itemize}
        \item Fehlersymptom: Wenn Einladungen vom Spieler angenommen werden, werden sie nicht von der Einladungsliste im Frontend entfernt
        \item Grund: Nicht implementiert
        \item Behebung: Einladungen von der Einladungsliste löschen und auf die Spiele Liste setzten
    \end{itemize}
    \subsection{Umlaute(Issue 37)}
    \begin{itemize}
        \item Fehlersymptom: Umlaute werden im Frontend nicht richtig dargestellt
        \item Grund: HTML Tag für UTF-8 nicht geschlossen und Ressourcebundle-Dateien können kein UTF-8 
        \item Behebung: Tag schließen und Oracle Unicode Konverter auf die Localisierung anwenden
    \end{itemize}
    \subsection{Skipping setzt Streak nicht zurück(Issue 56)}
    \begin{itemize}
        \item Fehlersymptom: Skipping einer Runde setzt die Streak eines Spielers nicht zurück
        \item Grund: API hat eine Exception geworfen bevor sie zum Skip Aufruf gekommen ist
        \item Behebung: Exception behoben
    \end{itemize}
\end{document}
