\documentclass[xcolor=dvipsnames]{beamer}
%
% Choose how your presentation looks.
%
% For more themes, color themes and font themes, see:
% http://deic.uab.es/~iblanes/beamer_gallery/index_by_theme.html
%
\mode<presentation>
{
  \usetheme{Darmstadt}      % or try Darmstadt, Madrid, Warsaw, ...
  \usecolortheme{wolverine} % or try albatross, beaver, crane, ...
  \usefonttheme{structurebold}  % or try serif, structurebold, ...
  \setbeamertemplate{navigation symbols}{}
  \setbeamertemplate{caption}[numbered]
  % eigenen stuff definieren
    % auf dieser website stehen weiter elemente von denen die farbe geändert werden kann
    % http://www.cpt.univ-mrs.fr/~masson/latex/Beamer-appearance-cheat-sheet.pdf
    % einfach ausprobieren was passt
    \definecolor{Grau}{HTML}{CCCCCC}
    \definecolor{GrauDark}{HTML}{777777} % auf weißem hintergrund
    \definecolor{Orange}{HTML}{EA5B10}
    \setbeamercolor{palette primary}{bg=Grau}
    \setbeamercolor{palette primary}{fg=BurntOrange}
    \setbeamercolor{normal text}{fg=GrauDark}
    \setbeamercolor{structure}{fg=BurntOrange} % farbe der items
    \setbeamertemplate{itemize item}[circle]
    \setbeamercolor{mini frame}{fg=BurntOrange}
    \setbeamercolor{section in head/foot}{bg=Grau}
    \setbeamercolor{section in head/foot}{fg=BurntOrange}
    \setbeamercolor{subsection in head/foot}{bg=white}
    \setbeamercolor{subsection in head/foot}{fg=GrauDark}
    \setbeamercolor{headline}{bg=Grau}
    \setbeamercolor{block body}{bg=white}
    \setbeamercolor{frametitle}{bg=white}
    
} 

\usepackage[utf8]{inputenc}
\usepackage[T1]{fontenc}
\usepackage{ae}
\usepackage{ngerman}
\usepackage{calc}
\usepackage{graphicx}

\title[Team 2 - Implementierung]{CS:Select Qualitätssicherung - Team 2}
\author{Luca Springer, Alexander Linder, Julian Dinh, Nicholas Bieker,\\ Bendix Sonnenberg}
\date{13.03.2019}

\begin{document}

\begin{frame} % das ist eine slide
  \titlepage
\end{frame}

\section{Kriterien}

\begin{frame}{Muss-Kriterien}

\end{frame}

\section{Schicke Implementierung}
\begin{frame}{Spieldarstellung}
\begin{itemize}
    \item Laden der Features passiert im Gameframe
    \item Aus der Rückgabe der API wird das Spiel gerendert
    \item Wenn das Spiel fertig ist, wird ein Event ausgelöst, was den Next-Button freischaltet
\end{itemize}
\end{frame}

\section{Änderungen zur Implementierung}
\begin{frame}{Database}
    \begin{itemize}
        \item Neue Oberklasse für alle Adapter
        \item Query-Klasse zur Bündelung von SQL-Statements
        \item PlayerStats-Adapter
        \item Auslagerung der Nutzerverwaltung in eigene Pakete
        \item FeatureSet Informationen verbleiben auf der Platte
    \end{itemize}
\end{frame}

\renewcommand{\arraystretch}{1.5}

\section{Statistiken}
\begin{frame}{Unit-Tests}
  \begin{center}
    \begin{tabular}{ | l | c | c | }
      \hline
      Paket & Testüberdeckung (\%) & Unit-Tests \\ \hline
      User & 60 & 16 \\
      Database & 62 & 44 \\
      Game & 74 & 45 \\
      Gamification & 96 & 34 \\
      Configuration & 76 & 10 \\
      \hline
    \end{tabular}
  \end{center}
\end{frame}

\begin{frame}{Lines of Code}
  \begin{itemize}
    \item Java: 7340 LOC, 2364 Kommentar-Zeilen \\
    \item JavaScript: 974 LOC \\
    \item JavaServer Pages: 405 LOC \\
    \item XML: 322 LOC \\
    \item Properties: 282 Zeilen \\
    \item CSS: 59 LOC \\
    \item Alle Dateien: 13765 Zeilen \\
  \end{itemize}
\end{frame}

\begin{frame}{GitHub - CS:Select}
    \begin{itemize}
        \item 570 Commits bis zur Abgabefrist \\
        \item 26 bearbeitete Issues in den letzten 2 Tagen vor Abgabefrist \\
    \end{itemize}
\end{frame}

\section{Docker}
\begin{frame}{Docker}
\begin{itemize}
    \item Docker Compose
    \item Container
    \begin{itemize}
        \item MySql
        \item Tomcat
        \item ML-Server
    \end{itemize}
    \item Konnte MySql Version leicht anpassen
\end{itemize}
\end{frame}

\section{Entwicklungsmodell}
    \begin{frame}{Kommunikation im Team}
        \begin{itemize}
            \item WhatsApp
            \item TeamSpeak
            \item Github Issues
        \end{itemize}
    \end{frame}
    \begin{frame}{Verwendung von Git}
        \begin{itemize}
            \item Feature Branches
            \item Pull Requests auf Master
            \item Code Review vor Merge
            \item Kleinere Anpassungen direkt auf Master
        \end{itemize}
    \end{frame}

\end{document}
